\section{LaTeXの使い方}
本章ではLaTeXの使い方をちょっとだけ解説します.
LaTeXは内容とスタイル(見た目)を切り分けて文書を編集することができるソフトウェアです.
コマンドを駆使して,美しい文書を作成することができます.

LaTeXの使い方については様々な書籍,ウェブサイトが解説を行っています.
詳しい使い方についてはそちらを参考にしてください.


\subsection{LaTeX環境}
\subsubsection{Overleaf}
OverLeafはオンライン上でLaTeXを執筆できる環境です.
自分のPC/Macの環境を汚さない,環境構築に苦労しないというメリットがあります.
一方,インターネットに接続していないと執筆作業ができないというデメリットがあります.

Overleafを使うと決めたら,\href{https://ja.overleaf.com}{OverLeaf}\footnote{https://ja.overleaf.com}のサイトに行きましょう.
そして,トップページの右上にある「登録する」をクリックしてアカウントを作成します.
アカウントの作成はGoogleアカウントを利用すると楽です.
研究室や大学で発行されたGoogleアカウントも使用可能です.

アカウントを作成したら,早速卒論用のプロジェクトを立ちあげましょう.
卒論用プロジェクトは山本がすでに作成してあります.
各人にOverleafから当該プロジェクトへの招待リンクを送ってあるので,そのリンクから卒論用プロジェクトを立ちあげてください.
問題がなければ,ブラウザ上にLaTeXの編集画面が表示されます.
画面左側にファイルブラウザ,およびその中に様々なファイルがあるのが確認できます.

最後にOverleafで日本語PDFを作成するための設定をします.
画面左上のメニューをクリックしてください.
表示されたメニューの中に「コンパイラ」という項目があるので,これを「LaTeX」に設定します.
さらに,「主要文書」を「\texttt{paper.tex}」に設定します.
以上で執筆環境の設定は完了です.

では早速OverleafでPDF文書を作成してみましょう.
ファイルブラウザから\texttt{DEIM}フォルダの中にある\texttt{paper.tex}を選びます.
クリックすると,\texttt{paper.tex}の中身が画面に表示されます.
この状態で,画面中央あたりにある「再編集」という緑色のボタンをクリックしましょう.
しばらくすると,LaTeXファイルがコンパイルされ,PDFが表示されます.

論文執筆環境が整いましたので,卒論投稿に向けて原稿を書き進めましょう.
\texttt{paper.tex}ファイルは論文のタイトルや著者名,および論文を構成する各章のコンテンツファイルへのリンクを記しただけのファイルです.
肝心の中身は\texttt{section}フォルダの中にあるファイルに書くように設定しています.
執筆対象となる章に対応するファイルをクリックし,中身を完成させてください.



\subsection{見出し}
文章を構造化するには,内容を章別,項別に整理することが重要です.
例えばこの文書では「第1章 LaTeXの使い方」が章に対応し,「1.1 LaTeX環境」が節に対応します.
LaTeXでは\texttt{section}コマンドを用いることで章見出しを,\texttt{subsection}コマンドを用いることで節見出しを作成することができます.
実際の使い方については,\texttt{section}ディレクトリにある\href{https://github.com/ymmt3-lab/thesis-template-2019/blob/master/section/how-to-use-latex.tex}{\texttt{how-to-use-latex.tex}ファイル}の中身を覗いてみてください.
本章に対応するLaTeXソースが確認できます.


\subsection{段落}
ある文章とある文章を段落で分けたい場合,LaTeXでは文章間に空行を入れることで段落を作ることができます.

この節の第2段落のLaTeXソースを確認してください.
前段落の文章とこの段落の文章との間に空行が設けられていることが確認できます.


\subsection{書体}
LaTeXではコマンドを用いて,文章の特定の箇所の書体を変更することができます.
例えばこの文書は地の文のデフォルト書体として明朝体が定義されていますが,{\tt gt}コマンドを用いることで,次の1文をゴシック体に変更することができます.
{\gt このように書体がゴシック体になりました.}
{\gt ゴシック体}以外にも,{\rm Roman famility},{\tt Typewriter familiy},{\sc SMALL CAPS SHAPE},など,様々な書体を利用することができます.
どういった書体がどのコマンドで利用できるかは,\href{http://www.latex-cmd.com/style/style.html}{「LaTeXコマンド集 - 書体」}等のウェブページで確認してください.


\subsection{文字装飾}
文字を太字にするには{\tt bf}コマンドを用います.
文字に下線を引くには{\tt underline}コマンドを用います.
これらのコマンドを使うことで,このように{\bf 太字}や\underline{下線}で文字を修飾することができます.
実際の使い方については,\texttt{section}ディレクトリにある\href{https://github.com/ymmt3-lab/DEIM-and-Thesis/blob/master/section/how-to-use-latex.tex}{\texttt{how-to-use-latex.tex}ファイル}の中身を覗いてみてください.


\subsection{箇条書き}
箇条書きは番号ありと番号なしの書き方ができます.
番号なしの箇条書きは,
\begin{itemize}
\item 箇条書き1
\item 箇条書き2
\item 箇条書き3
\end{itemize}
のような形式になります.
上記番号なし箇条書きを行うには,以下コードを書きます.
\begin{verbatim}
    \begin{itemize}
        \item 箇条書き1
        \item 箇条書き2
        \item 箇条書き3
    \end{itemize}
\end{verbatim}

番号ありの箇条書きは,
\begin{enumerate}
\item 箇条書きa
\item 箇条書きb
\item 箇条書きc
\end{enumerate}
のような形式になります.
上記番号あり箇条書きを行うには,以下コードを書きます.
\begin{verbatim}
    \begin{itemize}
        \item 箇条書き1
        \item 箇条書き2
        \item 箇条書き3
    \end{itemize}
\end{verbatim}


\subsection{図}

\begin{figure*}[tb]
\begin{center}
\includegraphics[width=13.5cm, clip]{figure/example.pdf}
\caption{図のキャプション}
\label{fig:example}
\end{center}
\end{figure*}

図を論文に掲載するには{\tt figure}コマンドおよび{\tt includegraphics}コマンドを用います.
具体的には以下のようなコマンドを書きます.

\begin{verbatim}
    \begin{figure*}[tb]
        \begin{center}
            \includegraphics[width=13.5cm, clip]{figure/example.pdf}
            \caption{図のキャプション}
            \label{fig:example}
        \end{center}
    \end{figure*}
\end{verbatim}
上記コマンドを使うことで,本節で掲載している図を掲載できます.
なお,{\tt figure}コマンドの代わりに{\tt figure*}コマンドを用いると,論文フォーマットが2カラムの場合,2カラム分のスペースを用いて図を掲載します.

LaTeXでは図の配置位置は大まかには指定できますが,ほぼ自動的に行われます.
指定できる位置としては主に「ページ上部(t)」「ページ下部(b)」「図コマンドを挿入した場所(h)」の3種類ありますが,学術論文ではページ上部もしくはページ下部のどちらかに図を配置することが一般的です.
図の配置位置は{\tt figure}コマンドのオプションで指定できます.
上の例では図の配置位置のページ上部(優先順位1),ページ下部(優先順位2)を指定します.

図を掲載するときは図にキャプションを添える必要があります.
キャプションは{\tt caption}コマンドを用います.

掲載した図(および表)は図表番号を使って論文中で参照します.
論文の修正過程で図表の掲載順を変えた場合,それに応じて本文中の図表参照番号を変更する必要があります(例:図\ref{fig:example}).
それを逐一手作業で行っていると面倒ですし,誤植の可能性も高まります
LaTeXではこの問題を解決するために{\tt label}コマンドを提供しています.
積極的に使いましょう.


\subsection{表}
表を論文に掲載するには{\tt figure}コマンドおよび{\tt table}コマンドを用います.
具体的には以下のようなコマンドを書きます.

\begin{verbatim}
    \begin{table}[tb]
      \begin{center}
        \caption{被験者の割り当て}
        \scalebox{1.00}{
        \begin{tabular}{c c c} \hline
            \toprule
                & \multicolumn{2}{c}{\textbf{性別}} \\
                \cmidrule(lr){2-3}
            \textbf{UI} & 男性 & 女性 \\
            \midrule
            Type A & 29 & 31 \\
            Type B & 26 & 32 \\
            \bottomrule
        \end{tabular}
        }
        \label{table:example}
      \end{center}
    \end{table}
\end{verbatim}
表の配置位置は{\tt figure}コマンドのオプションで指定できます.
上の例では表の配置位置のページ上部(優先順位1),ページ下部(優先順位2)を指定します.

\begin{table}[tb]
  \begin{center}
    \caption{被験者の割り当て.}
    \scalebox{1.00}{
    \begin{tabular}{c c c} \hline
        \toprule
            & \multicolumn{2}{c}{\textbf{性別}} \\
            \cmidrule(lr){2-3}
        \textbf{UI} & 男性 & 女性 \\
        \midrule
        Type A & 29 & 31 \\
        Type B & 26 & 32 \\
        \bottomrule
    \end{tabular}
    }
    \label{table:example}
  \end{center}
\end{table}
